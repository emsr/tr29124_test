\lstsetCPlusPlus
\begin{lstlisting}
e_float comp_ellint_1(const e_float& m);
e_float comp_ellint_2(const e_float& m);
e_float      ellint_1(const e_float& m, const e_float& phi);
e_float      ellint_2(const e_float& m, const e_float& phi);
\end{lstlisting}

\vspace{6.0pt}

\noindent {\it Effects:} These functions compute the complete elliptic integrals
and the incomplete elliptic integrals of the first and second kinds
for their respective arguments
{\courier phi}, {\courier m},
with {\courier phi}, {\courier m}~$\in\mathbb{R}$ and $|m| \leq 1$.

\vspace{6.0pt}

\noindent {\it Returns:} The {\courier comp\underline\ ellint\underline\ 1} function
returns~\cite{wolframfunctions:website}

\begin{equation}
K(m) = F(m, \pi/2)\quad .
\end{equation}

\noindent {\it Returns:} The {\courier comp\underline\ ellint\underline\ 2} function
returns~\cite{wolframfunctions:website}

\begin{equation}
E(m) = E(m, \pi/2)\quad .
\end{equation}

\noindent {\it Returns:} The {\courier ellint\underline\ 1} function
returns~\cite{wolframfunctions:website}

\begin{equation}
F(m, \phi) = \int_{0}^{\phi} \frac{d\theta}{\sqrt{1 - m \sin^2\theta}}\quad .
\end{equation}

\noindent {\it Returns:} The {\courier ellint\underline\ 2} function
returns~\cite{wolframfunctions:website}

\begin{equation}
E(m, \phi) = \int_{0}^{\phi} \sqrt{1 - m \sin^2\theta} \ d\theta\quad .
\end{equation}

\vspace{6.0pt}

\noindent {\it Remark:} There are several popular parameter conventions in use.
The naming convention with $|m| = k^2$ is used. The parameter naming convention
and the parameter order for the incomplete elliptic integrals in \efloat\
are identical to those used in the C++ special functions
draft \cite{isoiec29123:textbook}.

