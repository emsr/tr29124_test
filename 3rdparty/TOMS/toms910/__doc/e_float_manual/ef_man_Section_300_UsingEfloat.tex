Using and adapting the \efloat\ system is intuitive and straightforward.
Simple operations used for testing and benchmarking the \efloat\
library can be carried out by modifying the test file {\courier test.cpp}
and rebuilding \efloat, as described in Section \ref{sec:build} above.
Integration of \efloat\ in another independent project would, however,
require a separate build.

\subsection{Real--Numbered Arithmetic with \efloathyperref}

Using {\courier e\underline\ float} objects while coding is natural and
intuitive because {\courier e\underline\ float} objects can be used in
the same way as conventional plain--old--data (POD) floating-point data
types are used. Complete compatibility with the usual C++ semantics for
real--numbered arithmetic has been implemented.
There are also some extra utilities designed for standard situations
which commonly arise in numerical programming. For example, there is as a
globally defined digit tolerance called {\courier ef::tol(void)} within
the name\-space `{\courier ef}'. There is also a convenient
{\courier e\underline\ float} class member function called {\courier order},
which returns the base--$10$ `{big--O}' order of an
{\courier e\underline\ float} object.

The interface to real--numbered arithmetic with \efloat\ objects
is defined in the C++ header file
{\courier <e\underline\ float/e\underline\ float.h>}.
Real valued functions are defined within the name\-space `{\courier ef}'.
A complete synopsis of the \efloat\ class syntax is shown
in Chapter~\ref{chapter:classarch}.
There is no real valued arithmetic interface to any language other than C++.
There is no real valued arithmetic interface to the C language.
The code chunk below illustrates an example of non-trivial,
real valued arithmetic with \efloat\ by showing a possible implementation
of the small-argument Taylor series expansion of $\sin(x)$, $x\in\mathbb{R}$.
Common arithmetic operations and usage of some of the utility functions
are displayed.

\vspace{4.0pt}

\lstsetCPlusPlus
\lstinputlisting{CodeSample_ArithmeticReal.cpp}

\vspace{4.0pt}

The number of decimal digits is fixed at compile--time. The precision can be
dynamically changed during run--time for intermediate calculation steps,
but never increased to more than the fixed number of digits. The
number of digits is set by setting the value of
the preprocessor symbol {\courier E\underline\ FLOAT \underline\ DIGITS10}
which is used to initialize the value of
{\courier ef\underline\ \-digits10\underline\ \-setting},
which is a static constant 32-bit signed integer defined in the
public interface of \efloatbaseclass.

A specialization of the template class
{\courier std::\-nu\-me\-ric\underline\ lim\-its<e\underline\ float>}
has been defined using the usual semantics. This means that it is possible to
query details such as the number of decimal digits of precision or the
maximum \efloatclass\ value in the `usual' manner for the C++ language.
Support for formatted string output using {\courier std::o\-stream}
objects is defined using the usual semantics.
The output stream must be appropriately set
in order to display full precision. A code sample showing numeric limits,
setting output stream precision and printing to output stream is shown below.

\vspace{4.0pt}

\lstsetCPlusPlus
\lstinputlisting{CodeSample_OutputPrecision.cpp}

\vspace{4.0pt}

\subsection{Mixed--Mode Arithmetic with \efloathyperref}

All single--argument \efloatclass\ class constructors have been declared
with the {\courier explicit} qualifier. This prevents unwanted
side--effects such as the automatic compile--time conversion between
POD types and \efloatclass\ objects.
If, for example, the single--argument \efloatclass\ class constructor for
{\courier double} were non--explicit, then this

\vspace{4.0pt}

\lstsetCPlusPlus
\lstinputlisting{CodeSample_ArithmeticAmbiguousCtor.cpp}

\vspace{4.0pt}

\noindent would be ambiguous because the compiler would use conventional
double--precision for the sin function and subsequently convert to
an \efloatclass\ result. This would compromise the precision of the final
result because one of its intermediate calculation values would only have
the precision of {\courier double}.

However, even though all constructors \efloatclass\ class constructors
from POD types are explicit, there is support for global operators
for multiplication and division with constant {\courier INT32}.
These operations are exact and non--ambiguous because the representation
of {\courier INT32} is exact and non--ambiguous. Using the global
{\courier INT32} multiplication operator is shown in the code sample below.

\vspace{4.0pt}

\lstsetCPlusPlus
\lstinputlisting{CodeSample_ArithmeticMixedMode.cpp}

\vspace{4.0pt}

\noindent Remarking further on the code sample above, some users might expect
\efloatclass\ class constructors from integer types to be non--explicit
because the representation of integer types is exact and non--ambiguous.
However, the \efloatclass\ class constructors from integer types are
nonetheless declared explicit. Otherwise some common C++ code sequences would
lead to non--intuitive, confusing results. For example, this sequence

\vspace{4.0pt}

\lstsetCPlusPlus
\lstinputlisting{CodeSample_ArithmeticConfusingIfNonexplicitIntCtor.cpp}

\vspace{4.0pt}

\noindent might be expected to produce the value of $\frac{1}{7}$ to full precision.
This might be expected from the perspective of a symbolic system such as a
computer algebra system. But in fact, the C++ compiler actually reduces the value
of $\frac{1}{7}$ to $0$ in this case. The subsequent C++ initialization is
carried out with the pure--integer value $0$ instead of the full decimal
value of $\frac{1}{7}$. This situation is somewhat non--intuitive and confusing.
In order to facilitate non--ambiguity in these kinds of situations and to
avoid unexpected compiler side--efects, all \efloatclass\ constructors
from integer types are declared {\courier explicit}. Therefore when writing
code for these kinds of situations, the proper way to initialize fractions
such as $\frac{1}{7}$ or $\frac{1}{12}$ is like this

\vspace{4.0pt}

\lstsetCPlusPlus
\lstinputlisting{CodeSample_ArithmeticInitOneSeventh.cpp}

\vspace{4.0pt}

\noindent The strict use of the C++ style static cast could
be considered optional from a stylistic point of view.
Nonetheless, the author prefers to include the static cast
because this improves code portability by reducing
the degrees of freedom which exist for compiler interpretation
of the plain integer data type.


\subsection{Complex--Numbered Arithmetic with \efloathyperref}

Complex valued arithmetic is supported with the class {\courier ef\underline\ complex},
which has the same public interface as the STL template class
{\courier std::complex<{\it typename T}>}.
Complex valued functions are defined within the name\-space `{\courier efz}'.
The interface to complex valued arithmetic with {\courier ef\underline\ complex}
objects is defined in the C++ header file
{\courier <e\underline\ float/e\underline\ float\underline\ z.h>}.
Complex {\ttfamily ef\underline\ complex} objects can be used with each other
in normal arithmetic expressions and also mixed together with real
{\courier e\underline\ float} objects.
There is no complex valued arithmetic interface to any language other than C++.
There is no complex valued arithmetic interface to the C language.
The following code snippet displays the basic
use of {\ttfamily ef\underline\ complex} objects.

\vspace{4.0pt}

\lstsetCPlusPlus
\lstinputlisting{CodeSample_ArithmeticComplex.cpp}

\vspace{4.0pt}

\subsection{Using the \efloathyperref\ Functions}

\vspace{3.0pt}

Using the \efloat\ library of functions is straightforward. All of the
function prototypes are declared in C++ header files.
These have been collected in a single header file called
{\courier <func\-tions/\-func\-tions.h>} which contains
both real valued functions as well as complex valued
functions. The function prototypes are C++ function prototypes written in the
C++ language. There is no functional interface to any language other than C++.
There is no functional interface to the C language. A complete listing of the
functions supported by \efloat\ is provided in Section \ref{sec:functions}.
A use of functions is shown in the code snippet below.

\vspace{4.0pt}

\lstsetCPlusPlus
\lstinputlisting{CodeSample_Functions.cpp}

\vspace{4.0pt}

\subsection{Using the STL with \efloathyperref\ objects}

Real \efloat\ objects and complex
{\courier ef\underline\ complex} objects can be used
without limitation in standard STL containers and algorithms.
Both \efloat\ and {\courier ef\underline\ complex}
objects as well as pointers to these objects can be stored in all standard
STL and TR1 containers such as STL's template
{\courier std::\-vec\-tor<{\it typename T}>} class or TR1's template
{\courier std::\-tr1::\-array<{\it typename T}, {\courier std::\-size\underline\ t} {\it N}>}
class. This provides for powerful manipulation with STL algorithms such as
sorting, accumulating, streaming to output streams, etc. A use of
\efloat\ with a container, an algorithm, iterators, and stream output
is shown in the code sample below.

\vspace{4.0pt}

\lstsetCPlusPlus
\lstinputlisting{CodeSample_STL_Use.cpp}

\vspace{4.0pt}

\subsection{Additional Examples Using \efloathyperref}

Additional practical applications of \efloat\ are provided by seven
hand--written example files. The example files are stored in
{\courier{examples/example}}$\ast${\courier{.cpp}}, and they are
part of the tools block.

Example $1$ shows real--numbered usage in combination
with a timing measurement. It calculates $21$ non--trivial
values of $P_{\nu}^{\mu}(x)$, with $\nu$,$\,\mu$,$\,x\in\mathbb{R}$.
Example $2$ shows complex--num\-ber\-ed usage in combination
with a timing measurement. It calculates $21$ non--trivial
values of $\zeta(s)$, with $s\in\mathbb{Z}$.

Example $3$ shows mixed mode, real/integer operation by calculating the
real--num\-ber\-ed Jahnke--Emden--Lambda function,
$\Lambda_{\nu}(x)=\left\{\Gamma(\nu+1)\,J_{\nu}(x)\right\}\,/\,\left(\frac{1}{2}x\right)^{\nu}$
\cite{jahnkeemden:textbook}.
The small--argument series expansion employs arithmetic
operations as well as mixed--mode calculations.
It also shows how to effectively use STL containers of
{\courier{e{\ttfamily\underline\ }float}}
objects with STL algorithms.

Examples $4$ and $5$ use template utilities from the tools block.
Example $4$ performs a numerical differentiation,
$\frac{\partial}{\partial\nu}\,J_{\nu}(151+\gamma)\big|_{(\nu=123+C)}\,$,
where $\gamma$ is Euler's constant and $C$ is Catalan's constant.
The derivative central difference rule is ill--conditioned
and does not maintain full precision.
Example $5$ calculates a numerical integral,
$J_{0}(x)=\frac{1}{\pi}\int_{0}^{\pi}\,\cos(x\,\sin t)\,dt$, with $x=\left(12+\gamma\right)$
using a recursive trapezoid rule.
The utilities in Examples 4 and 5 make use of advanced
object--oriented templates. These examples show how static and dynamic
polymorphism can be combined to make efficient, elegant implementations
for standard numerical tasks.

In example~$6$, some well--known conventional algorithms are extended
to high precision and redesigned for use with real as well as complex
numbers. Luke~\cite{lukealgo:textbook} developed quadruple--precision
algorithms in For\-tran~$77$ which calculate coefficients for expansions
of hy\-per\-ge\-o\-me\-tric functions in series of Cheb\-y\-shev polynomials.
Luke's algorithms
{\courier{CCOEF2}} for $_{2}F_{1}(a, b; c; z)$ on page 59,
{\courier{CCOEF3}} for $_{1}F_{1}(a; b; z)$ on page 74,
and
{\courier{CCOEF6}} for $_{1}F_{2}(a; b, c; z)$ on page 85
have been extended to high precision. The inner loops have been
simplified through analyses with a computer algebra system.
Furthermore, these algorithms have been implemented as templates
for simultaneous use with real as well as complex pa\-ra\-me\-ters.
Table~\ref{table:functions} shows that \efloat's implementations
of hy\-per\-ge\-o\-me\-tric functions have strong limitations on their
parameter ranges. Example $6$ takes a promising first step toward
extending these parameter ranges.

Example~$7$ shows \efloat's interoperability with
the Math\-Link{\footnotesize {\textregistered}} facility of \mathematica.
Template programs combined with \efloat's
interface to computer algebra systems are used to
implement the generalized complex poly\-gam\-ma function
$\psi_{\nu}(x)$, with $\nu$,$\,x\in\mathbb{R}$ or $\mathbb{Z}$.
This extends the functionality of \efloat\ since its native al\-go\-rithm,
$\psi_{n}(x)$, only supports real parameters.

\subsection{Interoperability In Detail}

\subsubsection{Python~Export}
The \efloat\ export to Python produces a dynamic,
shared library. It is called
``{\courier{e{\ttfamily\underline\ }float{\ttfamily\underline\ }pyd.}}$\ast$'',
where the file ending will be either ``{\courier{pyd}}''
or ``{\courier{so}}'' for Windows{\footnotesize {\textregistered}} or
Unix/Linux--GNU systems respectively.
The shared library can be loaded into Python and the functions and classes of
\efloat\ can be used with its high level scripting capabilities
such as list manipulation. Interoperability with
mpmath~\cite{mpmath:website} can be achieved with Python strings.
The Python session below shows how to load the shared library and
generate a list of function values.
The final print command needs two carriage returns.

\lstset{language=python, basicstyle=\courierEight, keywordstyle=\ttfamily}
\begin{lstlisting}
>>> import e_float_pyd
>>> from e_float_pyd import(e_float, ef_complex, ef, efz)
>>> lst=[ef.cyl_bessel_j(ef.third(), k + ef.quarter()) for k in range(10)]
>>> for y in lst: print y
\end{lstlisting}

\subsubsection{Microsoft{\footnotesize {\textregistered}}~CLR~Export}
The \efloat\ export to the Microsoft{\footnotesize {\textregistered}} CLR
creates a CLR assembly for the Microsoft{\footnotesize {\textregistered}}.net Framework.
The name of the assembly is 
``{\courier{e{\ttfamily\underline\ }float{\ttfamily\underline\ }clr}}.{\courier{dll}}''.
It can be used with
all Microsoft{\footnotesize {\textregistered}} CLR
languages including C\#, managed C++/CLI, IronPython, {\emph{etc}}.
The code below illustrates the syntax of \efloat\ in the C\# language.

\lstset{language=[ISO]C++,basicstyle=\courierEight,commentstyle=\itshape,
keywordstyle=\bfseries,extendedchars=true}
\begin{lstlisting}
using e_float_clr.cli;
namespace test
{
  class Program
  {
    static void Main(string[] args)
    {
      e_float x = ef.half();
      e_float y = new e_float(123);
      ef_complex z = efz.riemann_zeta(new ef_complex(x, y));

      System.Console.WriteLine(z.get_str());
    }
  }
}
\end{lstlisting}


\subsubsection{Computer Algebra Systems}
Using a computer algebra system from within \efloat, in particular
extending \efloat\ with \mathematica, has
been discussed previously in Example~7.
The architecture for this is in the directory
{\courier{interop/cas}}. It uses an abstract base class called
{\courier{Com\-pu\-ter\-Al\-ge\-bra\-Sys\-tem\-Ob\-ject}} whose public interface
defines generic methods which exchange information with a
computer algebra system using STL strings and containers of \efloatclass\ objects.
The other interoperability direction,
using \efloat\ from within \mathematica, is not yet supported.
